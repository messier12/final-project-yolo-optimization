\begin{center}
  \Large
  \textbf{PREFACE}
\end{center}

\addcontentsline{toc}{chapter}{PREFACE}

\vspace{2ex}

% Ubah paragraf-paragraf berikut dengan isi dari kata pengantar

%Puji dan syukur kehadirat \lipsum[1][1-5]

%Penelitian ini disusun sebagai tugas akhir di Departemen Teknik Komputer - FTEIC - ITS.
%Penulis mengucapkan terima kasih kepada:
This research was made as a final project to fulfill the graduation requirement in Computer Engineering Department of ELECTICS - ITS.
The dataset used in this research is under CDLA-Permissive 1.0. 
Figures and tables reproduced in this report was either permissible under their copyright license or reproduced with the original author's permission.

I would like to thank both my advisors \advisor\ and \coadvisor\ who had given me guidance to complete this final project.
Also, I would like to thank the red computer in B201 lab, who has worked hard running gradient descent to train models for this project.
Last but not least, I would like to thank my friends who made working on this final project less boring.
%\begin{enumerate}[nolistsep]
%
%  %\item Keluarga, Ibu, Bapak dan Saudara tercinta yang telah \lipsum[3][1-2]
%
%  \item Para dosen pembimbing yang telah memberi arahan jalannya 
%
%  \item \lipsum[5][1-3]
%
%\end{enumerate}
%
%Akhir kata, semoga \lipsum[6][1-8]

\begin{flushright}
  \begin{tabular}[b]{c}
    \place{}, \ENGMONTH{} \the\year{} \\
    \\
    \\
    \\
    \\
    \name{}
  \end{tabular}
\end{flushright}
