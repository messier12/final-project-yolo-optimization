\begin{center}
  \large
  \textbf{OPTIMISASI PENDETEKSIAN OBJEK KECIL YOLOv7 UNTUK MENDETEKSI OBJEK-OBJEK \emph{AIRBORNE}}
  % \textbf{OPTIMISASI YOLOv7 UNTUK PENDETEKSIAN OBJEK KECIL BERUPA OBJEK \emph{AIRBORNE}}
  % YOLOv7 small object detection optimization to detect airborne objects 
  % Optimisasi Pendeteksian Objek Kecil YOLOv7 untuk mendeteksi objek airborne
\end{center}
\addcontentsline{toc}{chapter}{ABSTRAK}
% Menyembunyikan nomor halaman
\thispagestyle{empty}

\begin{flushleft}
  \setlength{\tabcolsep}{0pt}
  \bfseries
  \begin{tabular}{l@{\hspace{2pt}}l@{\hspace{6pt}}l}
  Nama Mahasiswa / NRP&:& Dion Andreas Solang / 07211940000039\\
  Departemen&:& Teknik Komputer FTEIC - ITS\\
  Dosen Pembimbing&:& 1. Reza Fuad Rachmadi, S.T., M.T., Ph.D\\
  & & 2. Dr. I Ketut Eddy Purnama S.T., M.T.\\
  \end{tabular}
  \vspace{4ex}
\end{flushleft}
\textbf{Abstrak}

% Isi Abstrak
Objek-objek \emph{airborne} merupakan objek-objek yang akan terlihat sangat kecil pada kamera.
YOLOv7 merupakan model pendeteksi objek \emph{real time state of the art} yang dioptimisasi untuk pendeteksian objek umum.
Oleh karena itu, untuk mendeteksi objek \emph{airborne} dengan baik, perlu dilakukan modifikasi pada YOLOv7.
Tujuan dari penelitian ini adalah untuk menemukan solusi modifikasi YOLOv7 yang dapat mengoptimisasi kemampuan pendeteksian objek kecil khususnya objek \emph{airborne}.
Modifikasi yang dilakukan terhadap YOLOv7 meliputi modifikasi arsitektur dan modifikasi \emph{bag-of-freebies}.
Modifikasi arsitektur meliputi modifikasi \emph{neck} dan penambahan \emph{layer head}. 
Modifikasi \emph{bag-of-freebies} meliputi penambahan augmentasi mosaik dan rekalkulasi \emph{anchor} aktif.
Modifikasi-modifikasi ini akan dikombinasikan dan diuji performanya.
Modifikasi yang menghasilkan model dengan skor mAP tertinggi pada dataset \emph{airborne} akan dipilih sebagai solusi optimisasi pada penelitian ini.


%Abstrak harus berisi seratus hingga dua ratus kata. \lipsum[1]

\vspace{2ex}
\noindent
\textbf{Kata Kunci: \emph{Deteksi Objek Kecil, YOLOv7, Modifikasi Arsitektur, Modifikasi Bag-of-Freebies, Objek Airborne}}

%\vfill
%\vspace{4ex}
%
%\begin{minipage}[t]{0.5\textwidth}
%  \centering
%  \vspace{3ex}
%  Dosen Pembimbing 1\\
%  \vspace{6em}
%  \underline{Reza Fuad Rachmadi, S.T. M.T., Ph.D}\\
%  NIP: 19850403201212 1 001\\
%\end{minipage}%
%\begin{minipage}[t]{0.5\textwidth}
%  \centering
%  Surabaya, 14 Februari 2022\\
%  \vspace{2ex}
%  Dosen Pembimbing 1\\
%  \vspace{6em}
%  \underline{Reza Fuad Rachmadi, S.T. M.T., Ph.D}\\
%  NIP: 19850403201212 1 001\\
%\end{minipage}
%
%\vspace{2em}
%\begin{center}
%Mengetahui,\\
%Departemen Teknik Komputer FTEIC-ITS\\
%Kepala,\\
%\vspace{6em}
%\underline{Reza Fuad Rachmadi, S.T. M.T., Ph.D}\\
%NIP: 19850403201212 1 001\\
%\end{center}
%