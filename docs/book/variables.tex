% Atur variabel berikut sesuai namanya

% nama
\newcommand{\name}{Dion Andreas Solang}
\newcommand{\authorname}{Solang, Dion Andreas}
\newcommand{\nickname}{Dion}
\newcommand{\advisor}{Reza Fuad Rachmadi, S.T., M.T., Ph.D}
\newcommand{\coadvisor}{Dr. I Ketut Eddy Purnama, S.T., M.T.}
\newcommand{\examinerone}{Dr. Eko Mulyanto Yuniarno,S.T.,M.T.}
\newcommand{\examinertwo}{Ahmad Zaini, S.T., M.Sc.}
\newcommand{\examinerthree}{}
\newcommand{\headofdepartment}{Dr. Supeno Mardi Susiki Nugroho, S.T., M.T}

% identitas
\newcommand{\nrp}{0721 19 4000 0039}
  %\textmd{NIP } \\
\newcommand{\advisornip}{19850403201212 1 001}
\newcommand{\coadvisornip}{19690730199512 1 001}
\newcommand{\examineronenip}{19680601199512 1 009}
\newcommand{\examinertwonip}{19750419200212 1 003}
\newcommand{\examinerthreenip}{19540925197803 1 001}
\newcommand{\headofdepartmentnip}{19700313199512 1 001}

% judul
\newcommand{\tatitle}{OPTIMISASI PENDETEKSIAN OBJEK KECIL YOLOv7 UNTUK MENDETEKSI OBJEK-OBJEK \emph{AIRBORNE}}
\newcommand{\engtatitle}{\emph{YOLOv7 SMALL OBJECT DETECTION OPTIMIZATION TO DETECT AIRBORNE OBJECTS}}

% tempat
\newcommand{\place}{Surabaya}

% jurusan
\newcommand{\studyprogram}{Teknik Komputer}
\newcommand{\engstudyprogram}{Computer Engineering}

% fakultas
\newcommand{\faculty}{Fakultas Teknologi Elektro dan Informatika Cerdas}
\newcommand{\engfaculty}{Faculty of Intelligent Electrical and Informatics Technology}

% singkatan fakultas
\newcommand{\facultyshort}{FTEIC}
\newcommand{\engfacultyshort}{ELECTICS}

% departemen
\newcommand{\department}{Teknik Komputer}
\newcommand{\engdepartment}{Computer Engineering}

% kode mata kuliah
\newcommand{\coursecode}{EC184801}
