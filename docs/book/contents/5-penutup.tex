\chapter{CONCLUSION}

% Ubah bagian-bagian berikut dengan isi dari penutup

\section{Conclusion}
\label{section:Conclusion}

Based on the experiments conducted in previous chapter, we can conclude that:
among the modification candidates proposed in this research, we found the combination
of mosaic augmentation, anchor recalculation, and rerouting feature map from P3 to P2
produced the greatest mAP@50 score 14.09\%. 
Another improvement we made by partitioning in the input image and perform inference 
on each of them independently. We find that a YOLOv7 model with mosaic augmentation,
and replacement of head layer with decoupled anchor-free head gives out the greatest
mAP@50 score of 46.18\%. 
%Berdasarkan hasil pengujian yang dilakukan, dapat disimpulkan bahwa:
%Dari himpunan modifikasi-modifikasi yang diajukan pada penelitian ini,
%didapatkan kombinasi modifikasi yang paling optimal.
%Modifikasi tersebut penambahan augmentasi mosaik, rekalkulasi anchor, dan pemindahan koneksi neck-backbone.
%Dengan modifikasi itu, skor mAP@50 dari YOLOv7 meningkat sebanyak 14,09\%.

%\section{Discussion}
%Modifikasi-modifikasi yang diaplikasikan pada penelitian ini hingga saat ini
%masih dibatasi pada modifikasi yang tidak secara signifikan mempengaruhi \emph{latency}.
%Beberapa modifikasi seperti melakukan partisi pada gambar, dan melakukan deteksi di
%tiap partisinya. Dengan metode ini, objek yang dipelajari akan terlihat lebih besar
%sehingga lebih mudah untuk dideteksi. Akan tetapi, latency pendeteksian untuk tiap gambar
%akan meningkat sebanyak jumlah partisi kali lipat. Untuk mengatasi hal tersebut,
%kita dapat melakukan down-scaling pada model yang digunakan. Modifikasi jenis ini
%perlu dicoba untuk menentukan jumlah partisi dan down-scaling model yang tepat
%untuk mempertahankan kecepatan deteksi yang \emph{real-time}.
%\section{Discussion}
%With greater computaational resource, this research could probably be better.



