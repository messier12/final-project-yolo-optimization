\chapter{CONCLUSION}

% Ubah bagian-bagian berikut dengan isi dari penutup

\section{Conclusion}
\label{section:conclusion}

Based on the experiments conducted in previous chapter, it can be observed that some modification
can improve YOLOv7 ability on detecting airborne objects, while others have negligible or even negative effect.
We found that:
\begin{itemize}[noitemsep,topsep=0pt]
  \item YOLOv7 was only able to perform detection on airborne objects after adding mosaic augmentation to the training data and recalculated its anchor to fit the data.
  This modification improved the mAP score from 0 to 11.2\%.
  \item EIoU, even with its convexication technique, was not able to outperform YOLOv7's original CIoU. It produced only 4.92\% in mAP score.
  \item Rerouting the detection to use an earlier stage of feature map can improve slightly improve the mAP.
  \item Additional head on YOLOv7 did not perform better than original 3 head.
  \item Replacing head to a decoupled anchor free head greatly improve the detection capability.
  \item Applying image partition to all model greatly improve their detection performance. The greatest performance after applying image partition
  came from anchor-free head modification which produced 46.18\% mAP, a 13.2\% improvement compared to without partition.
\end{itemize}

In conclusion, amongst the proposed modifications applied to YOLOv7, we found that applying image partition, and replacing
head to a decoupled anchor free head performed the best on detecting airborne objects, while still able to perform inference in real-time.


%\section{Discussion}

%Based on the experiments conducted in previous chapter, we can conclude that:
%among the modification candidates proposed in this research, we found the combination
%of mosaic augmentation, anchor recalculation, and rerouting feature map from P3 to P2
%produced the greatest mAP@50 score 14.09\%. 
%Another improvement we made by partitioning in the input image and perform inference 
%on each of them independently. We find that a YOLOv7 model with mosaic augmentation,
%and replacement of head layer with decoupled anchor-free head gives out the greatest
%mAP@50 score of 46.18\%. 
%Berdasarkan hasil pengujian yang dilakukan, dapat disimpulkan bahwa:
%Dari himpunan modifikasi-modifikasi yang diajukan pada penelitian ini,
%didapatkan kombinasi modifikasi yang paling optimal.
%Modifikasi tersebut penambahan augmentasi mosaik, rekalkulasi anchor, dan pemindahan koneksi neck-backbone.
%Dengan modifikasi itu, skor mAP@50 dari YOLOv7 meningkat sebanyak 14,09\%.

%\section{Discussion}
%Modifikasi-modifikasi yang diaplikasikan pada penelitian ini hingga saat ini
%masih dibatasi pada modifikasi yang tidak secara signifikan mempengaruhi \emph{latency}.
%Beberapa modifikasi seperti melakukan partisi pada gambar, dan melakukan deteksi di
%tiap partisinya. Dengan metode ini, objek yang dipelajari akan terlihat lebih besar
%sehingga lebih mudah untuk dideteksi. Akan tetapi, latency pendeteksian untuk tiap gambar
%akan meningkat sebanyak jumlah partisi kali lipat. Untuk mengatasi hal tersebut,
%kita dapat melakukan down-scaling pada model yang digunakan. Modifikasi jenis ini
%perlu dicoba untuk menentukan jumlah partisi dan down-scaling model yang tepat
%untuk mempertahankan kecepatan deteksi yang \emph{real-time}.
%\section{Discussion}
%With greater computaational resource, this research could probably be better.



