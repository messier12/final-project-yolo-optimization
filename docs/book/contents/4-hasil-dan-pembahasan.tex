\chapter{HASIL DAN PEMBAHASAN}

% Ubah bagian-bagian berikut dengan isi dari pengujian dan analisis

Pada bab ini, akan dipaparkan pengaruh modifikasi-modifikasi yang dilakukan pada YOLOv7.

\section{Performa Awal}
Untuk mengukur pengaruh dari modifikasi-modifikasi yang dilakukan pada YOLOv7, maka
hal pertama yang harus dilakukan adalah mengukur performa YOLOv7 tanpa segala modifikasi
yang diajukan pada bab \ref{section:modificationcandidates}. Arsitektur YOLOv7 \emph{plain} 
ini di-\emph{train} pada 400 sampel data dari  \textcite{aot_dataset} dengan 300 epoch dan batch size 1.
Dengan aturan tersebut, ditemukan bahwa model \emph{plain} tidak mampu untuk mendeteksi
objek apapun pada dataset uji, dengan kriteria "terdeteksi" $IoU > 0.5$ (mAP@.5 = 0).

Untuk keperluan komparasi dengan performa-performa dari modifikasi pada YOLOv7,
model \emph{plain} ini akan selanjutnya disebut sebagai \verb*|YOLOv7-plain|.

\section{Pengaruh Augmentasi Mosaic dan Rekalkulasi Anchor}


Terdapat 3 modifikasi yang diujikan pada bagian ini, yaitu \verb*|YOLOv7-plain| yang ditambahkan augmentasi mosaic,
\verb*|YOLOv7-plain| yang direkalkulasi anchor, dan \verb*|YOLOv7-plain| yang ditambahkan augmentasi mosaic dan rekalkulasi anchor.
\subsection{Augmentasi Mosaic}
  Proses melakukan augmentasi mosaic cukup \emph{straightforward},
  augmentasi ini dilakukan pada beberapa data training.
  Contoh hasil augmentasi dapat dilihat pada gambar \ref{fig:mosaic-train}
  \begin{figure}[H]
    \centering
    \includegraphics[scale=0.4]{figures/mosaic-aug-2.png}
    \caption{Contoh Augmentasi Mosaic pada Dataset Training}
    \label{fig:mosaic-train}
  \end{figure}

  %, sedangkan untuk rekalkulasi anchor akan dibahas pada bagian berikut.
\subsection{Rekalkulasi Anchor}
  Rekalkulasi anchor dilakukan dengan mengklaster data training ke 9 centroid menggunakan algoritma k-means.
  Sembilan centroid tersebut digunakan sebagai anchor, 3 untuk tiap head pada arsitektur YOLO (terdapat 3 head).
  Persebaran anchor sebelum dan sesudah direkalkulasi dapat dilihat 
  pada Tabel \ref{tbl:recalculated_anchor} dan Gambar \ref{fig:anchor-dist}

    \begin{tabular}{ c l l }
    \toprule[1.5pt]
    Head       & Original Anchors        & Recalculated Anchors\\
    \midrule
    1          & [12,16], [19,36], [40,28]& [4,4], [14,5], [11,11]\\
    2          & [36,75], [76,55], [72,146]& [27,8], [18,17], [41,14]\\
    3          & [142,110], [192,243], [459,401]& [43,32], [86,27], [149,70]\\
    \bottomrule[1.5pt]
  \end{tabular}
  \vspace{1ex}
  \begin{figure}[H]
    \centering
    \includegraphics[width=\textwidth]{figures/anchor-dist-2.png}
    \caption{Persebaran Anchor. Kiri: Anchor Lama. Kanan: Anchor Hasil Rekalkulasi}
    \label{fig:anchor-dist}
  \end{figure}

  Jika kita memperhatikan persebaran anchor sebelum dan sesudah direkalkulasi pada Gambar \ref{fig:anchor-dist},
  dapat kita lihat bahwa anchor hasil rekalkulasi lebih mencakup seluruh persebaran dataset daripada anchor lama.
  8 dari 9 anchor lama bertempat di kuadran pertama dari garis median(garis putus-putus).
  Hal ini berarti 8 anchor tersebut hanya mampu mendeteksi sekitar 25\% dari objek-objek pada dataset.
  Sedangkan, anchor hasil rekalkulasi menempatkan anchor di setiap kuadran.



\subsection{Performa Augmentasi Mosaik dan Rekalkulasi Anchor}
  Performa dari tiap modifikasi dapat dilihat pada tabel \ref{tbl:mosaic_reanchor_performance}.
  Pada tabel tersebut, terlihat bahwa YOLOv7 mampu untuk mendeteksi beberapa objek pada dataset uji ketika diberi 
  augmentasi mosaic pada data train dan direkalkulasi anchornya.
  \begin{table}[H]
  \centering
  \captionof{table}{Performa Modifikasi Augmentasi Mosaic dan Rekalkulasi Anchor}
  \label{tbl:mosaic_reanchor_performance}
  \vspace{-1ex}
  \begin{tabular}{ l l c }
    \toprule[1.5pt]
    No & Modifikasi        &mAP@50 \\
    \midrule
    0  & \textbf{yolov7-plain               }& 0\%\\
    1  & \textbf{mosaic                     }& 0\%\\
    2  & \textbf{rekalkulasi anchor         }& 0\%\\
    3  & \textbf{mosaic + rekalkulasi anchor}& \textbf{11,2}\%\\
    \midrule
       & Peningkatan                         & \textbf{\textcolor{green}{+11,2\%}}\\
    \bottomrule[1.5pt]
  \end{tabular}
\end{table}

  Hanya modifikasi nomor 3 yang mampu melakukan deteksi, maka 
  modifikasi tersebut dijadikan baseline untuk modifikasi-modifikasi lainnya.
  Untuk mempermudah komparasi penambahan modifikasi-modifikasi selanjutnya,
  maka model ini akan disebut sebagai \verb*|YOLOv7-base|.

\section{Pengaruh Penggantian \emph{Box Loss Function}}
Dengan menggunakan \verb*|YOLOv7-base| sebagai baseline, 
Box Loss function dari YOLOv7 diganti menjadi EIoU.
Telah juga dilakukan percobaan menggunakan convexciation pada EIoU.
Hasil dari pengujian dapat dilihat pada Tabel \ref{tbl:loss_function_perf}

  \begin{tabular}{ l l c }
    \toprule[1.5pt]
    No & Modification        &mAP@50 \\
    \midrule
    0  & \texttt{\textbf{YOLOv7-MAR +CIoU}} (original)     & \textbf{11.2}\%\\
    1  & \texttt{YOLOv7-MAR + EIoU}                & 0\%\\
    2  & \texttt{YOLOv7-MAR + EIoU + Convexication} & 4.92\%\\
    \midrule
       & Improvement                                & \textbf{\textcolor{orange}{-6.28\%}}\\
    \bottomrule[1.5pt]
  \end{tabular}

Ternyata, meskipun EIoU memiliki performa lebih baik daripada CIoU ketika diaplikasikan
pada Faster-RCNN+ResNet50 dengan dataset VOC2007 dan COCO2017, EIoU tidak mampu untuk meningkatkan
AP deteksi YOLOv7 pada dataset \textcite{aot_dataset}.

\section{Pengaruh Perubahan Koneksi \emph{Neck-Backbone}}
Untuk modifikasi ini, koneksi \emph{neck-backbone} yang diubah adalah
koneksi layer neck yang memberikan feature pada \emph{head} pertama yang
awalnya terkoneksi dengan skala 8 dari backbone, dipindahkan ke skala 4.
Hal ini diilustrasikan pada Gambar \ref{fig:deeperconn}.

\begin{figure}[H]
  \centering
  \includegraphics[width=0.7\textwidth]{figures/deeperconn.png}
  \caption{Modifikasi Koneksi Neck. Kiri : Sebelum. Kanan : Sesudah.}
  \label{fig:deeperconn}
\end{figure}
  \begin{tabular}{ l l c }
    \toprule[1.5pt]
    No & Modifikasi                                      &mAP@50 \\
    \midrule
    0  & \texttt{YOLOv7-MAR}                           & 11.2\%\\
    1  & \texttt{\textbf{YOLOv7-MAR + rerouting}} & \textbf{14.09\%}\\
    \midrule
       & Improvement                                & \textbf{\textcolor{green}{+2.98\%}}\\
    \bottomrule[1.5pt]
  \end{tabular}
Perbandingan performa modifikasi ini dengan \verb*|YOLOv7-base| dapat dilihat pada tabel \ref{tbl:neck_backbone_perf}.
Terlihat bahwa modifikasi ini berhasil meningkatkan skor mAP@50
dari \verb*|YOLOv7-base| sebesar 2,98\%. Untuk mempermudah perbandingan dengan modifikasi lain, model hasil modifikasi
ini akan disebut \verb*|YOLOv7-moveconnection|
\vspace{2ex}

\section{Pengaruh Penambahan \emph{Head}}
Untuk modifikasi ini, pada skala 4 backbone, dipasangkan suatu layer head tambahan.
Ilustrasi penambahan layer ini dapat dilihat pada gambar \ref{fig:addinghead}.
\begin{figure}[H]
  \centering
  \includegraphics[width=0.35\textwidth]{figures/addhead.png}
  \caption{Penambahan Layer Head}
  \label{fig:addinghead}
\end{figure}

  \begin{tabular}{ l l c }
    \toprule[1.5pt]
    No & Modification                                &mAP@50 \\
    \midrule
    0  & \texttt{YOLOv7-MAR}                        & \textbf{11.2\%}\\
    1  & \texttt{YOLOv7-MAR + more head}            & 5.19\%\\
    \midrule
       & Improvement                                & \textbf{\textcolor{orange}{-6\%}}\\
    \bottomrule[1.5pt]
  \end{tabular}
Seperti yang dapat dilihat pada tabel \ref{tbl:addhead}, penambahan modifikasi ini memberi performa 
yang lebih buruk dibandingkan \verb*|YOLOv7-base|. 
Padahal, modifikasi penambahan layer head dan \verb*|YOLOv7-moveconnection| dua-duanya menggunakan fitur pada skala 4.
Alasan untuk hal ini akan diinvestiagsi dengan melihat output dari tiap skala pada model penambahan head dan \verb*|YOLOv7-moveconnection|.

\section{Pengaruh Pengubahan \emph{Head} menjadi \emph{Anchor-Free}}
Penggantian \emph{head} menjadi \emph{decoupled anchor-free head} membuat model tidak mampu mendeteksi apapun pada dataset uji (mAP=0\%).
\begin{tabular}{ l l c }
  \toprule[1.5pt]
  No & Model                                 &mAP@50 \\
  \midrule
  0  & \texttt{YOLOv7-MAR (anchor head)}        & 11.2\%\\
  1  & \texttt{YOLOv7-MAR + anchor-free head}       & 0\% \\
  1  & \textbf{\texttt{YOLOv7-MAR + anchor-free head}} \textsuperscript{*numerically stabilized}      & \textbf{32.98\%} \\
  \midrule
     & Improvement      & \textbf{\textcolor{green}{+21.78\%}} \\
  \bottomrule[1.5pt]
\end{tabular}

