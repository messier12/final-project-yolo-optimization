\begin{center}
  \large\textbf{ABSTRAK}
\end{center}

\addcontentsline{toc}{chapter}{ABSTRAK}

\vspace{2ex}

\begingroup
% Menghilangkan padding
\setlength{\tabcolsep}{0pt}

\noindent
\begin{tabularx}{\textwidth}{l >{\centering}m{2em} X}
  Nama Mahasiswa    & : & \name{}         \\

  Judul Tugas Akhir & : & \tatitle{}      \\

  Pembimbing        & : & 1. \advisor{}   \\
                    &   & 2. \coadvisor{} \\
\end{tabularx}
\endgroup

% Ubah paragraf berikut dengan abstrak dari tugas akhir
%Pada penelitian ini kami mengajukan \lipsum[1]
Pada penelitian ini, kami menunjukan percobaan kami untuk meningkatkan kapabilitas YOLOv7
untuk mendeteksi objek airborne. Objek airborne tampak sangat kecil pada kamera karena
jarak yang cukup jauh dari kamera. Oleh karena itu, YOLOv7 harus dioptimisasi untuk
dapat mendeteksi objek-objek kecil. Beberapa modifikasi diajukan dan diuji pada penelitian
ini. Modifikasi-modifikasi ini meliput perubahan arsitektur (Menambah layer deteksi, mengubah
sumber feature-map, dan mengganti layer deteksi menjadi anchor-free), dan pada bag-of-freebies
(rekalkulasi anchor, dan augmentasi mosaik). Hingga saat ini, kami menemukan bahwa kombinasi
augmentasi mosaik, rekalkulasi anchor, dan mengubah sumber feature-map memberikan skor
mAP yang paling tinggi 14,09\% dibandingkan dengan YOLOv7 yang biasa (mAP=0\%).

% Ubah kata-kata berikut dengan kata kunci dari tugas akhir
\noindent
\textbf{Kata Kunci: \emph{Deteksi Objek Kecil, YOLOv7, Modifikasi Arsitektur, Modifikasi Bag-of-Freebies, Objek Airborne}}
