\begin{center}
  \large\textbf{ABSTRAK}
\end{center}

\addcontentsline{toc}{chapter}{ABSTRAK}

\vspace{2ex}

\begingroup
% Menghilangkan padding
\setlength{\tabcolsep}{0pt}

\noindent
\begin{tabularx}{\textwidth}{l >{\centering}m{2em} X}
  Nama Mahasiswa    & : & \name{}         \\

  Judul Tugas Akhir & : & \tatitle{}      \\

  Pembimbing        & : & 1. \advisor{}   \\
                    &   & 2. \coadvisor{} \\
\end{tabularx}
\endgroup

% Ubah paragraf berikut dengan abstrak dari tugas akhir
%Pada penelitian ini kami mengajukan \lipsum[1]
%Pada penelitian ini, kami menunjukan percobaan kami untuk meningkatkan kapabilitas YOLOv7
%untuk mendeteksi objek airborne. Objek airborne tampak sangat kecil pada kamera karena
%jarak yang cukup jauh dari kamera. Oleh karena itu, YOLOv7 harus dioptimisasi untuk
%dapat mendeteksi objek-objek kecil. Beberapa modifikasi diajukan dan diuji pada penelitian
%ini. Modifikasi-modifikasi ini meliput perubahan arsitektur (Menambah layer deteksi, mengubah
%sumber feature-map, dan mengganti layer deteksi menjadi anchor-free), dan pada bag-of-freebies
%(rekalkulasi anchor, dan augmentasi mosaik). Hingga saat ini, kami menemukan bahwa kombinasi
%augmentasi mosaik, rekalkulasi anchor, dan mengubah sumber feature-map memberikan skor
%mAP yang paling tinggi 14,09\% dibandingkan dengan YOLOv7 yang biasa (mAP=0\%).

%Pada penelitian ini, kami menunjukan percobaan kami untk meningkatkan kapabilitas dari YOLOv7 untuk mendeteksi objek airborne.
%Objek airborne tampak 
Pada penelitian ini, kami menunjukan percobaan kami untuk meningkatkan kemampuan deteksi YOLOv7 terhadap objek \emph{airborne}. 
Objek-objek di \emph{airborne} tampak sangat kecil pada gambar kamera ketika berada dalam jarak yang jauh. Namun, karena kecepatan pergerakan objek \emph{airborne} itu tinggi, penting untuk mendeteksinya saat masih berada dalam jarak yang jauh. 
Oleh karena itu, YOLOv7 perlu dioptimalkan untuk dapat mendeteksi objek-objek kecil dengan baik. 
Dalam penelitian ini, kami mengusulkan beberapa modifikasi yang berupa perubahan dalam arsitektur (menambahkan kepala deteksi tambahan, mengalihkan skala fitur deteksi, dan mengganti \emph{head} YOLO dengan \emph{decoupled anchor-free head}), penerapan teknik \emph{bag-of-freebies} (rekalkulasi anchor dan augmentasi mosaik), serta merubah proses inferensi (mempartisi gambar dan melakukan inferensi pada setiap partisi). 
Melalui eksperimen yang komprehensif, kami menemukan bahwa kombinasi penggantian \emph{head} YOLO dengan \emph{decoupled anchor-free head} dan melakukan inferensi pada partisi-partisi menghasilkan model yang memiliki peningkatan paling signifikan pada \emph{mean average precision} (mAP) yaitu sebesar 46,18\% dan tetap mempertahankan kecepatan inferensi yang \emph{real-time} (>10 FPS). 
Peningkatan ini jauh lebih tinggi dibandingkan dengan YOLOv7 polos tanpa modifikasi yang hanya mampu mencapai skor mAP sebesar 0\%.

% Ubah kata-kata berikut dengan kata kunci dari tugas akhir
\noindent
\textbf{Kata Kunci: \emph{Deteksi Objek Kecil, YOLOv7, Modifikasi Arsitektur, Modifikasi Bag-of-Freebies, Objek Airborne}}
