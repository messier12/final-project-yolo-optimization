\begin{center}
  \large\textbf{ABSTRACT}
\end{center}

\addcontentsline{toc}{chapter}{ABSTRACT}

\vspace{2ex}

\begingroup
% Menghilangkan padding
\setlength{\tabcolsep}{0pt}

\noindent
\begin{tabularx}{\textwidth}{l >{\centering}m{3em} X}
  \emph{Name}     & : & \name{}         \\

  \emph{Title}    & : & \engtatitle{}   \\

  \emph{Advisors} & : & 1. \advisor{}   \\
                  &   & 2. \coadvisor{} \\
\end{tabularx}
\endgroup

% Ubah paragraf berikut dengan abstrak dari tugas akhir dalam Bahasa Inggris
%\emph{In this research, we proposed \lipsum[1]}
{
  \itshape
  In this research, we present an attempt to improve the detection capability of YOLOv7 for airborne objects. 
  Airborne objects appear considerably small in camera images when they are located at a considerable distance from the camera. 
  However, due to their high speed of movement, it is crucial to detect them while they are still far away. 
  Therefore, to effectively detect these objects, YOLOv7 needs to be optimized for small objects.
  To address this challenge, we proposed several modifications that include changes in the architecture (adding an extra detection head, modifying the feature-map source, and replacing the detection head with a detached anchor-free head), application of bag-of-freebies techniques (anchor recalculation and mosaic augmentation), and change in the inference process (partitioning the image and performing inference on each partition).
  Through comprehensive experimentation, we have discovered that the combination of replacing the detection head with a detached anchor-free head, and performing inference on partitions yields the most promising results, with a significant increase in mean average precision (mAP) of 46.18\% while still maintaining real-time inference speed (greater than 10 FPS).
  This improvement is notably higher compared to the unmodified plain YOLOv7, which achieved a mAP score of 0\%.
}
%{
%\itshape
%In this research, we present an attempt to improve the capability of YOLOv7 to detect
%airborne objects. Airborne objects appear very small on cameras when they have large distance
%from the camera. However, they also move really fast, making it important to detect them while they are still far away. 
%For that reason, YOLOv7 needs to be optimized to detect small objects.
%Several modification proposals was made and tested in this research. These modifications
%include changes in the architecture (Adding extra detection head, modifying feature-map
%source, and replacing detection head to a detached anchor-free head), some bag-of-freebies
%applications (anchor recalculation, mosaic augmentation), and changes in the way the neural network perform inference (partition the image and do inference on each partition). 
%We found that the combination of replacing detection head to detached anchor-free head and performing inference on partitions
%produce a model with the greatest mAP score 46.18\% that still maintains inference speed in real-time (> 10 FPS). 
%This increase is significantly higher compared to YOLOv7 plain without any modifications applied that yield mAP of 0\%.
%%mosaic augmentation, anchor recalculation, and modifying feature-map
%%source produces the greatest score in mAP, a 14.09\% increase compared to the plain
%%YOLOv7 (mAP=0\%).
%}

% Ubah kata-kata berikut dengan kata kunci dari tugas akhir dalam Bahasa Inggris
%\emph{Keywords}: \emph{}, \emph{Anti-gravity}, \emph{Energy}, \emph{Space}.
\noindent
\textbf{Keywords: \emph{Small Object Detection, YOLOv7, Architecture Modification, Bag-of-Freebies Modification, Airborne Object}}
