\begin{center}
  \large\textbf{ABSTRACT}
\end{center}

\addcontentsline{toc}{chapter}{ABSTRACT}

\vspace{2ex}

\begingroup
% Menghilangkan padding
\setlength{\tabcolsep}{0pt}

\noindent
\begin{tabularx}{\textwidth}{l >{\centering}m{3em} X}
  \emph{Name}     & : & \name{}         \\

  \emph{Title}    & : & \engtatitle{}   \\

  \emph{Advisors} & : & 1. \advisor{}   \\
                  &   & 2. \coadvisor{} \\
\end{tabularx}
\endgroup

% Ubah paragraf berikut dengan abstrak dari tugas akhir dalam Bahasa Inggris
%\emph{In this research, we proposed \lipsum[1]}
{
\itshape
In this research, we present an attempt to improve the capability of YOLOv7 to detect
airborne objects. Airborne objects appear very small on cameras due to their distance
to the camera. For that reason, YOLOv7 needs to be optimized to detect small objects.
Several modification proposal was made and tested in this research. These modifications
include changes in the architecture (Adding extra detection head, modifying feature-map
source, and replacing detection head to a detached anchor-free head), and some bag-of-freebies
applications (anchor recalculation, mosaic augmentation). At the current state, we found
the combination of mosaic augmentation, anchor recalculation, and modifying feature-map
source produces the greatest score in mAP, a 14.09\% increase compared to the plain
YOLOv7 (mAP=0\%).
}

% Ubah kata-kata berikut dengan kata kunci dari tugas akhir dalam Bahasa Inggris
%\emph{Keywords}: \emph{}, \emph{Anti-gravity}, \emph{Energy}, \emph{Space}.
\noindent
\textbf{Keywords: \emph{Small Object Detection, YOLOv7, Architecture Modification, Bag-of-Freebies Modification, Airborne Object}}
