\begin{center}
  \large
  \textbf{YOLOv7 SMALL OBJECT DETECTION OPTIMIZATION TO DETECT AIRBORNE OBJECTS}
  % \textbf{YOLOv7 OPTIMIZATION FOR SMALL OBJECT DETECTION TO DETECT AIRBORNE OBJECTS}
  % Optimisasi YOLOv7 untuk pendeteksian objek kecil berupa objek airborne
  % YOLOv7 small object detection optimization to detect airborne objects 
  % Optimisasi Pendeteksian Objek Kecil YOLOv7 untuk mendeteksi objek airborne
\end{center}
% Menyembunyikan nomor halaman
\thispagestyle{empty}

\begin{flushleft}
  \setlength{\tabcolsep}{0pt}
  \bfseries
  \begin{tabular}{lc@{\hspace{6pt}}l}
  Student Name / NRP&:& Dion Andreas Solang / 07211940000039\\
  Department&:& Teknik Komputer FTEIC - ITS\\
  Advisors&:& 1. Reza Fuad Rachmadi, S.T., M.T., Ph.D\\
  & & 2. Dr. I Ketut Eddy Purnama S.T., M.T.\\
  \end{tabular}
  \vspace{4ex}
\end{flushleft}
\textbf{Abstract}

% Isi Abstrak
Airborne objects appear very small on cameras.
YOLOv7 is the state of the art real-time object detector optimized for general object detections.
Thus, to detect airborne objects with YOLOv7, modifications are needed to be applied.
The purpose of this research is to find a modification solution for YOLOv7 to optimize its small object detection capability especially for airborne objects.
Modifications to be applied on YOLOv7 consists of architecture modifications and bag-of-freebies modifications.
Architecture modifications consist of neck modification and head layer addition.
Bag-of-freebies modifications consist of mosaic data augmentation dan active anchor recalculation.
These modifications will be combined one with another and have their performance tested.
Modifications that produces model with the highest mAP score on airborne objects dataset will be chosen as the optimization solution of this research.
%The abstract must consist between two hundred to three hundred words. \lipsum[1]

\vspace{2ex}
\noindent
\textbf{Keywords: \emph{Small Object Detection, YOLOv7, Architecture Modification, Bag-of-Freebies Modification, Airborne Object}}