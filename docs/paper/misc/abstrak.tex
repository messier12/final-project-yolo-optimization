\begin{center}
  \large
  \textbf{OPTIMISASI PENDETEKSIAN OBJEK KECIL YOLOv7 UNTUK MENDETEKSI OBJEK-OBJEK \emph{AIRBORNE}}
  % \textbf{OPTIMISASI YOLOv7 UNTUK PENDETEKSIAN OBJEK KECIL BERUPA OBJEK \emph{AIRBORNE}}
  % YOLOv7 small object detection optimization to detect airborne objects 
  % Optimisasi Pendeteksian Objek Kecil YOLOv7 untuk mendeteksi objek airborne
\end{center}
\addcontentsline{toc}{chapter}{ABSTRAK}
% Menyembunyikan nomor halaman
\thispagestyle{empty}

\begin{flushleft}
  \setlength{\tabcolsep}{0pt}
  \bfseries
  \begin{tabular}{l@{\hspace{2pt}}l@{\hspace{6pt}}l}
  Nama Mahasiswa / NRP&:& Dion Andreas Solang / 07211940000039\\
  Departemen&:& Teknik Komputer FTEIC - ITS\\
  Dosen Pembimbing&:& 1. Reza Fuad Rachmadi, S.T., M.T., Ph.D\\
  & & 2. Dr. I Ketut Eddy Purnama S.T., M.T.\\
  \end{tabular}
  \vspace{4ex}
\end{flushleft}
\textbf{Abstrak}

% Isi Abstrak
Abstrak harus berisi seratus hingga dua ratus kata. \lipsum[1]

\vspace{2ex}
\noindent
\textbf{Kata Kunci: \emph{Roket, Anti-gravitasi, Meong}}