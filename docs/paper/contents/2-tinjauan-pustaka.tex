\chapter{TINJAUAN PUSTAKA}
\section{Teori Dasar}
  \subsection{Arsitektur Famili YOLO}
    Arsitektur famili YOLO pada dasarnya terbagi akan 3 bagian yaitu \emph{head}, \emph{neck}, dan \emph{backbone}.
    Setiap bagian ini mempunyai fungsi masing-masing.
    Berikut adalah penjelasan fungsi dan cara kerja dari ketiga bagian tersebut.
    \subsection{\emph{Layer Head} YOLO dan \emph{Anchor Box}}
      \begin{figure}[ht]
          \centering
          \includegraphics[scale=0.4]{pictures/anchorbox.png}
          \caption{Prediksi \emph{Anchor Box} dan \emph{offset} dari koordinat latis \parencite{yolov3}}
          \label{fig:anchorbox}
      \end{figure}
      Arsitektur famili YOLO yang dipublikasikan setelah YOLOv2 terus menggunakan \emph{anchor box} untuk melakukan deteksi \parencites{yolov2}{yolov3}{yolov4}{scaledyolov4}{yolov5}{yolor}{yolov7}.
      \emph{Anchor boxes} merupakan beberapa \emph{Bounding Box} yang telah terdefinisikan. 
      Arsitektur YOLO akan memprediksi probabilitas \emph{anchor box} berada pada suatu koordinat latis beserta dengan \emph{offset anchor box} tersebut untuk menepatkan \emph{anchor box} pada objek yang dideteksi.
      Penggunaan \emph{anchor box} ini dapat meningkatkan akurasi deteksi karena \emph{neural network} hanya perlu mencari titik tengah objek dan \emph{error} dimensi \emph{boudning box} dengan menggunakan \emph{offset} \parencite{yolov3}.
      Hal ini lebih sederhana daripada mencari titik-titik \emph{bounding box} secara independen sehingga lebih mudah untuk dipelajari oleh \emph{neural network}.
  
      Prediksi \emph{bounding boxes} terjadi di bagian \emph{head} dari arsitektur YOLO.
      Bagian \emph{head} dari YOLO akan mengambil beberapa hasil \emph{upsampling} yang terjadi pada \emph{neck} YOLO, dan kemudian melakukan prediksi \emph{anchor boxes} dari hasil tersebut.
      Hasil prediksi \emph{Head} YOLO pada suatu tingkatan \emph{upsampling} berupa tensor dengan ukuran $N\times N \times [A\times(4+1+C)]$ dengan $N$ sebagai dimensi hasil \emph{upsampling}-nya, $A$ sebagai jumlah \emph{anchor boxes} untuk \emph{scaling} tersebut, dan $C$ sebagai jumlah kelas prediksi.
      Angka 4 merepresentasikan 4 \emph{offset} $b_x, b_y, b_w, b_h$ seperti pada Gambar \ref{fig:anchorbox} dan angka 1 merepresentasikan \emph{objectness score} dari prediksi \emph{bounding box}.
  
    \subsection{\emph{Neck} YOLO}
  
      \begin{figure}[ht]
          \centering
          \includegraphics[scale=0.6]{pictures/yolo-architecture-rough.png}
          \caption{\emph{Feature Pyramid Network} pada \emph{Neck} YOLO}
          \label{fig:yolofpn}
      \end{figure}
  
      \emph{Neck} dari YOLO merupakan \emph{layer-layer} dimana \emph{head} YOLO mengambil fitur untuk dilakukan deteksi \emph{bounding box}.
      Pada YOLOv3 \textcite{yolov3}, arsitektur \emph{neck} dibuat menyerupai \emph{Feature Pyramid Network} (FPN) seperti pada Gambar \ref{fig:yolofpn}. 
      Pada versi-versi YOLO selanjutnya, bentuk \emph{neck} ini tidak banyak berubah dan pada dasarnya tetap mempertahankan bentuk \emph{pyramid}-nya.
  
      Penaikkan tingkatan \emph{pyramid} dari FPN merupakan \emph{upsampling} dari \emph{feature map} yang dihasilkan \emph{backbone}.
      Output tiap tingkatan pada FPN di \emph{neck} inilah yang diinputkan pada \emph{head} YOLO. 
      Melakukan prediksi pada tingkatan \emph{upsampling} yang berbeda-beda dapat membuat \emph{neural network} mendapatkan lebih banyak informasi semantik dan informasi yang lebih detail sehingga dapat lebih akurat dalam mendeteksi objek besar maupun kecil.
  
    \subsection{\emph{Backbone} YOLO}
      \emph{Backbone} dari YOLO merupakan bagian yang mengekstrak fitur dari citra yang diinputkan.
      Hasil ekstraksi fitur ini akan diinputkan pada \emph{neck} yang kemudian akan di\emph{upsampling} olehnya.
      Model-model YOLO dapat menggunakan \emph{feature extractor} dari model-model klasifikasi citra sebagai \emph{backbone}-nya.
      Sebagai contoh, salah satu varian YOLO, YOLO-Z menggunakan DenseNet sebagai \emph{backbone}-nya sedangkan arsitektur YOLO dasarnya, YOLOv5 menggunakan \emph{backbone} YOLOv5v7.0 \parencite{yoloz}.
  
  
  
  
  \subsection{YOLOv7}
    YOLOv7 merupakan pendeteksi objek \emph{real time} dengan skor akurasi tertinggi pada dataset COCO di tahun 2022.
    Pada YOLOv7, dilakukan beberapa perubahan untuk meningkatkan akurasi dan kecepatan deteksinya.
    Perubahan-perubahan tersebut dilakukan pada arsitekturnya dan pada \emph{bag-of-freebies}-nya.
  
    Perubahan arsitektur dilakukan pada \emph{backbone}. YOLOv7 menggunakan \emph{Extended Efficient Layer Aggregation Network} (E-ELAN) sebagai \emph{backbone}, berbeda dengan leluhurnya YOLOv4 yang menggunakan CSP-Darknet.
    E-ELAN merupakan arsitektur \emph{neural network} yang efisien karena E-ELAN didesain dengan mengontrol \emph{gradient path} terpanjang yang terpendek.
    Karena efisiensinya, arsitektur E-ELAN ini dapat meningkatkan kecepatan deteksi dan akurasi. \parencite{yolov7}
  
    \emph{Bag-of-freebies} merupakan kumpulan metode peningkatan akurasi yang tidak meningkatkan \emph{cost inferrence} \parencite{yolov4}. 
    Pada YOLOv7, ditambahkan beberapa \emph{bag-of-freebies} yang dapat dilatih seperti \emph{re-parameterized convolution} dan \emph{extra auxilary head} di tengah-tengah \emph{neural network}.
    Selain kedua itu, YOLOv7 juga menambahkan \emph{trainable bag-of-freebies} dari YOLOR seperti EMA, \emph{Implicit Knowledge}, dan \emph{conv-bn topology Batch Normalization} \parencite{yolov7}.


\section{Rekalkulasi \emph{Anchor}}
  \emph{Anchor box} dari model-model \emph{pre-trained} YOLO pada umumnya mengoptimisasi \emph{anchor box} modelnya pada dataset COCO.
  Ukuran \emph{anchor box} yang akan digunakan pada model YOLO dapat dikonfigurasikan agar lebih sesuai dengan dataset yang akan digunakan untuk melatih model YOLO.
  Penyesuaian ini dapat meningkatkan IoU(\emph{Intersection Over Union}) prediksi model dengan \emph{ground truth} sehingga meningkatkan akurasi.

  Penyesuaian dapat dilakukan dengan cara mengkonfigurasi secara manual tiap ukuran \emph{anchor box} atau dengan menggunakan algoritma \emph{clustering}.
  Penggunaan algoritma \emph{clustering} akan lebih baik karena setiap ukuran \emph{anchor box}-nya disesuaikan dengan pengelompokan-pengelompokan ukuran \emph{bounding box} natural yang terdapat pada dataset.

\section{Augmentasi Mosaik}
  \begin{figure}[ht]
    \centering
    \includegraphics[scale=0.6]{pictures/mosaic-aug.png}
    \caption{Contoh Augmentasi Data Mosaik \parencite{yolov5}}
    \label{fig:mosaic}
  \end{figure}

  Augmentasi mosaik merupakan teknik augmentasi yang baru dikenalkan pada YOLOv4.
  Teknik augmentasi ini akan memilih 4 gambar dari dataset, memotong gambar-gambar tersebut dan menggabungkannya secara acak pada satu gambar seperti pada Gambar \ref{fig:mosaic}.
  Hasil dari penggabungan itu membuat gambar terlihat seperti mosaik.
  Teknik augmentasi ini mampu meningkatkan akurasi model \parencite{yolov4}.


\section{Penelitian Terkait}
  \subsection{YOLO-Z}
    YOLO-Z merupakan arsitektur famili YOLO yang modifikasi dari YOLOv5 \parencite{yoloz}.
    Modifikasi-modifikasi yang dilakukan meliputi pergantian \emph{backbone}, \emph{neck}, dan jumlah \emph{anchor}
    \emph{Backbone} dari YOLOv5r5.0 menjadi DenseNet yang di-\emph{downscale}.
    \emph{Neck} dari YOLO-Z juga diganti dari PanNet menjadi FPN dan biFPN tergatung pada \emph{scale} YOLO-Z yang digunakan.

    Modifikasi pada YOLO-Z didesain untuk mendeteksi objek kecil untuk tujuan melakukan deteksi \emph{cone} yang nampak jauh pada lintasan \emph{autonomous racing} secara \emph{real time} (lihat Gambar \ref{fig:yolozcone}).
    Modifikasi-modifikasi dibuktikan dapat meningkatkan kemampuan pendeteksian objek kecil \parencite{yoloz}.
    Oleh karena itu, untuk meningkatkan kemampuan mendeteksi objek kecil YOLOv7, beberapa modifikasi yang dilakukan YOLO-Z pada YOLOv5 dapat diaplikasikan.
    \begin{figure}[ht]
      \centering
      \includegraphics[scale=0.4]{pictures/yoloz-cone.png}
      \caption{Contoh Objek \emph{Cone} yang Terlihat Jauh dari Kamera}
      \label{fig:yolozcone}
    \end{figure}

  \subsection{exYOLO}
    exYOLO merupakan hasil modifikasi arsitektur YOLOv3 \parencite{exyolo}.
    Pada exYOLO, dilakukan modifikasi \emph{neck} dengan menambahkan suatu \emph{Receptive Field Block} sebelum penggabungan \emph{feature map} yang akan diupsampling.
    Modifikasi-modifikasi ini membuat exYOLO memiliki skor mAP yang lebih tinggi daripada YOLOv3 pada dataset PASCAL VOC 2007.